\documentclass[12pt,mathserif]{beamer}
\usepackage{color}

\definecolor{titleblue}{rgb}{0.2,0.2,0.7}
\newcommand{\bl}[1]{\color{titleblue}{#1 }\color{black}}
\newcommand{\gr}[1]{\color{green1}{#1 }\color{black}}
\newcommand{\rd}[1]{\color{red}{#1 }\color{black}}
\newcommand{\bk}[1]{\color{black}{#1 }\color{black}}
\newcommand{\wh}[1]{\color{white}{#1 }\color{black}}

\setbeamertemplate{navigation symbols}{} %no nav symbols
\usepackage{textcomp} % this package has arrows (see comprehensive LaTeX symbol list)
\usepackage[NoDocJS]{pdfanim}
\PDFAnimLoad[height=3in]{fishanim}{anim1/fishanim}{19}

\title{How to make informative figures, \\ purty pictures, and waste time with \\ free and open source software}
\author{Ian Taylor}
\institute{Quantitative Ecology and Resource Management\\
University of Washington}
\date{May 7, 2008}


%%%%%%%%%%%%%%%%%%%%%%%%%%%%%%%%%%%%%%%%%%%%%%%%%%%%%%%%%
\begin{document}

\frame{
  \tableofcontents[part=1]
    \maketitle
}

\begin{frame}[fragile]
\begin{center}\includegraphics[width=3in]{flowchart.pdf}\end{center}
\end{frame}

% \section{\LaTeX \textrightarrow inkscape \textrightarrow R}
\begin{frame}[fragile]
{\LaTeX packages {Ti\emph{k}Z} \& PGF}
\hspace{2in}\includegraphics[width=1.5in]{flowchart.pdf}
\scriptsize
\vspace{-.3in}
\begin{verbatim}
\begin{tikzpicture}[auto,bend left]
  \node (R)     at   (0:1)  {\includegraphics[width=.2in]{Rlogo.jpg}};
  \node (latex) at (120:1)  {\LaTeX};
  \node [text width=1cm](ink) at (240:1){Inkscape
                          \includegraphics[width=.2in]{inkscape.logo.pdf}};

  \path (R)     edge[very thick,<->] (Inkscape)
        (ink)   edge[very thick,<->] (latex)
        (latex) edge[very thick,<->] (R);
\end{tikzpicture}
\end{verbatim}
\end{frame}

\begin{frame}
{Inkscape}
\includegraphics[width=1in]{ink.pdf}
\begin{itemize}
\item Inkscape is a vector graphics editor application.
\item ``Vector graphics is the use of geometrical primitives such as points, lines, curves, and shapes or polygon(s), which are all based upon mathematical equations, to represent images in computer graphics.''\footnote{\url{http://en.wikipedia.org/wiki/Vector_graphics}}
\item Two key features I'm aware of so far:
\begin{itemize}
\item edit PDF
\item trace bitmap.\\
\end{itemize}
\end{itemize}
\end{frame}

\begin{frame}
% {Inkscape: edit PDF to adjust plot from R}
{\includegraphics[height=.2in]{Rlogo.jpg} \textrightarrow \includegraphics[height=.2in]{ink2.pdf} \textrightarrow \bk{\textrm{\LaTeX}} \hfill \hfill \hfill \bl{edit PDF}}
\begin{center}\includegraphics[height=6in]{/home/ian/dogshare/dissertation/ianphd/metapopulation/figs/boxSensDeplNew.pdf}
\end{center}\end{frame}

\begin{frame}
{\includegraphics[height=.2in]{Rlogo.jpg} \textrightarrow \includegraphics[height=.2in]{ink2.pdf} \textrightarrow \bk{\textrm{\LaTeX}} \hfill \hfill \hfill \bl{edit PDF}}
\begin{center}\includegraphics[height=6in]{/home/ian/dogshare/dissertation/ianphd/metapopulation/figs/boxSensDeplNewFixed.pdf}
\end{center}\end{frame}


\begin{frame}
{\includegraphics[height=.2in]{ink2.pdf} \textrightarrow \bk{\textrm{\LaTeX}} \hfill \hfill \hfill \bl{trace bitmap, add text}}
\vspace{-.2in}\begin{center}\includegraphics[height=3.1in]{eli1.png}
\end{center}\end{frame}


\begin{frame}
{\includegraphics[height=.2in]{ink2.pdf} \textrightarrow \bk{\textrm{\LaTeX}} \hfill \hfill \hfill \bl{trace bitmap, add text}}
\begin{center}\includegraphics[height=3in]{eli7.pdf}
\end{center}\end{frame}


\begin{frame}
{\includegraphics[height=.2in]{ink2.pdf} \textrightarrow \includegraphics[height=.2in]{Rlogo.jpg} \hfill \hfill \hfill \bl{trace bitmap, export}}
\begin{center}\includegraphics[height=3in]{eli8.pdf}
\end{center}\end{frame}


\begin{frame}
{\includegraphics[height=.2in]{ink2.pdf} \textrightarrow \includegraphics[height=.2in]{Rlogo.jpg} \hfill \hfill \hfill \bl{trace bitmap, export}}
\vspace{-0.5in}\begin{center}\includegraphics[height=3in]{eli9.pdf}
\pause
\end{center}
\vspace{-0.8in} ``Vector graphics is the use of geometrical primitives such as points, lines, curves, and shapes or \rd{polygon(s), which are all based upon mathematical equations,} to represent images in computer graphics.''
\end{frame}


\begin{frame}[fragile]
\tiny
\begin{verbatim}
%LaTeX with PSTricks extensions
%%Creator: inkscape 0.46
%%Please note this file requires PSTricks extensions
%\psset{xunit=.5pt,yunit=.5pt,runit=.5pt}
\begin{pspicture}(200,150)
{
  \newrgbcolor{curcolor}{0.20784314 0.24705882 0.81176472}
  \pscustom[linestyle=none,fillstyle=solid,fillcolor=curcolor]
  {
    \newpath
    \moveto(62.481523,45.57681)
    \curveto(58.004483,52.025789)(53.961043,54.497499)(49.314352,53.625779)
    \curveto(45.893402,52.984009)(42.555682,55.26854)(44.389722,56.99648)
    \curveto(44.888912,57.46679)(51.594992,58.38924)(59.292123,59.04637)
    \curveto(66.989253,59.70349)(73.636733,60.45735)(74.064303,60.7216)
    \curveto(74.491873,60.98585)(76.565533,63.53184)(78.672443,66.37934)
    \curveto(80.779343,69.22684)(84.447673,73.44041)(86.824283,75.74284)
    \lineto(91.145383,79.92908)
    \lineto(87.858773,82.99922)
    \curveto(84.764683,85.88954)(84.612583,86.34723)(85.262193,90.81299)
    \curveto(85.641703,93.42198)(86.525463,96.14864)(87.226113,96.87223)
    \curveto(87.971723,97.64226)(91.145193,98.26432)(94.878463,98.37223)
    \lineto(101.25692,98.55661)
    \lineto(102.39403,95.10098)
    \curveto(103.80439,90.81498)(105.30864,90.23153)(107.61264,93.07685)
    \curveto(109.3492,95.2214)(110.90552,95.93584)(123.29734,100.27701)
    \curveto(126.59734,101.43308)(130.64734,103.38913)(132.29734,104.62379)
    \curveto(136.41234,107.70295)(147.05384,108.47965)(147.60432,105.74101)
    %...six lines cut out...
    \curveto(101.09076,54.99221)(99.030263,54.62606)(86.797343,54.117669)
    \lineto(73.297343,53.556609)
    \lineto(72.126123,49.55661)
    \curveto(71.481953,47.35661)(70.685873,44.31911)(70.357073,42.80661)
    \curveto(69.424843,38.5184)(66.721533,39.46928)(62.481523,45.57681)
    \closepath
  }
}
\end{pspicture}
\end{verbatim}
\end{frame}


\begin{frame}[fragile]
\tiny
\begin{verbatim}







# R code
fish = c(
    62.481523,45.57681,
    58.004483,52.025789,53.961043,54.497499,49.314352,53.625779,
    45.893402,52.984009,42.555682,55.26854,44.389722,56.99648,
    44.888912,57.46679,51.594992,58.38924,59.292123,59.04637,
    66.989253,59.70349,73.636733,60.45735,74.064303,60.7216,
    74.491873,60.98585,76.565533,63.53184,78.672443,66.37934,
    80.779343,69.22684,84.447673,73.44041,86.824283,75.74284,
    91.145383,79.92908,
    87.858773,82.99922,
    84.764683,85.88954,84.612583,86.34723,85.262193,90.81299,
    85.641703,93.42198,86.525463,96.14864,87.226113,96.87223,
    87.971723,97.64226,91.145193,98.26432,94.878463,98.37223,
    101.25692,98.55661,
    102.39403,95.10098,
    103.80439,90.81498,105.30864,90.23153,107.61264,93.07685,
    109.3492,95.2214,110.90552,95.93584,123.29734,100.27701,
    126.59734,101.43308,130.64734,103.38913,132.29734,104.62379,
    136.41234,107.70295,147.05384,108.47965,147.60432,105.74101,
    #...six lines cut out...
    101.09076,54.99221,99.030263,54.62606,86.797343,54.117669,
    73.297343,53.556609,
    72.126123,49.55661,
    71.481953,47.35661,70.685873,44.31911,70.357073,42.80661,
    69.424843,38.5184,66.721533,39.46928,62.481523,45.57681)

fish = matrix(fish,length(fish)/2,2,byrow=T)
\end{verbatim}
\end{frame}


\begin{frame}
{\includegraphics[height=.2in]{ink2.pdf} \textrightarrow \includegraphics[height=.2in]{Rlogo.jpg} \hfill \hfill \hfill \bl{trace bitmap, export}}
\begin{center}\includegraphics[height=3in]{eli9.pdf}
\end{center}
\end{frame}


\begin{frame}
{\includegraphics[height=.2in]{ink2.pdf} \textrightarrow \includegraphics[height=.2in]{Rlogo.jpg} \hfill \hfill \hfill \bl{trace bitmap, export}}
\begin{center}\includegraphics[height=3in]{fish1.pdf}
\end{center}
\end{frame}

\begin{frame}
{\includegraphics[height=.2in]{ink2.pdf} \textrightarrow \includegraphics[height=.2in]{Rlogo.jpg} \hfill \hfill \hfill \bl{trace bitmap, export}}

{\large Rotation matrices:}\footnote{\url{http://en.wikipedia.org/wiki/Rotation_matrix}}

In matrix theory, a rotation matrix\ldots is a real special orthogonal matrix. The name refers to the fact that an $n \times n$ rotation matrix corresponds to a geometric rotation about a fixed origin in an n-dimensional Euclidean space\ldots.
\[Q_{2 \times 2} = \begin{bmatrix}\cos \theta & -\sin \theta \\ \sin \theta & \cos \theta\end{bmatrix},\]
\end{frame}

\begin{frame}
{\includegraphics[height=.2in]{ink2.pdf} \textrightarrow \includegraphics[height=.2in]{Rlogo.jpg} \hfill \hfill \hfill \bl{trace bitmap, export}}
\begin{center}
\PDFAnimation{fishanim}
\end{center}
\end{frame}

\setbeamertemplate{background canvas}{\includegraphics[width=\paperwidth]{fish2.pdf}}
% \begin{frame}[plain]

\begin{frame}
{\includegraphics[height=.2in]{ink2.pdf} \textrightarrow \includegraphics[height=.2in]{Rlogo.jpg} \hfill \hfill \hfill \bl{trace bitmap, export}}
\begin{center}

\end{center}
\end{frame}
\setbeamertemplate{background canvas}{} %return to blank canvas


\begin{frame}
{\includegraphics[height=.4in]{ubuntu_logo.png}}
\begin{columns}
\column{.3\textwidth}
{Currently most popular desktop linux distribution. \\ \ \\ All open-source software extremely easy to install.}
\column{.7\textwidth}
\includegraphics[width=\textwidth]{Screenshot-3.png}
\end{columns}
\begin{columns}
\column{.2\textwidth}
\includegraphics[width=\textwidth]{knuth.jpg}
\column{.7\textwidth}
``I currently use Ubuntu Linux''\\ Donald Knuth (creater of \TeX), \\ April 25, 2008
\end{columns}
\end{frame}

\begin{frame}
{Books of Edward Tufte}
One-day courses in Seattle July 17 \& 18, 2008,\\ \$200 with student ID (includes copies of 4 books): \\ \url{http://www.edwardtufte.com/}.

\includegraphics[width=.23\textwidth]{cover_vdqi.png}
\includegraphics[width=.23\textwidth]{cover_ei.png}
\includegraphics[width=.23\textwidth]{cover_visex.png}
\hspace{.1in}
\includegraphics[width=.25\textwidth]{cover_be.png}

\end{frame}


\end{document}