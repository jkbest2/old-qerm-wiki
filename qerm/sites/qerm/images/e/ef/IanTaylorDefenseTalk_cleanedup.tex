\documentclass[12pt,mathserif]{beamer}
\usepackage{color}
%\usepackage{multimedia}
\usepackage{float}
\usepackage{hyperref}
\usepackage{movie15}

\usepackage[orientation=landscape,size=custom,width=16,height=9,scale=0.5,debug]{beamerposter}
% \usepackage[orientation=landscape,size=custom,width=171,height=96,scale=1.0,debug]{beamerposter}

\setbeamertemplate{navigation symbols}{} %no nav symbols

\hypersetup{
    pdfpagetransition=Glitter,
    pdftitle={Ian's talk Spring 2007},
    pdfsubject={},
    pdfauthor={Ian},
    pdfkeywords={dogfish},
    pdfpagemode=None,
%    plainpages=false,
    pdfstartview=Fit,
%    breaklinks=true,
%    colorlinks=false,
%    pdfhighlight=/N,
    % define colors, even if not used
%    linkcolor=blue,
%    citecolor=blue,
%    urlcolor=blue,
%    citebordercolor=1 1 1,
%    filebordercolor=1 1 1,
%    linkbordercolor=1 1 1,
%    menubordercolor=1 1 1,
%    pagebordercolor=1 1 1,
%    urlbordercolor=1 1 1,
%    pdfborder=1 1 1
  }
% load the pdfanim style, should be done after hyperref
% NOT JUST ``SHOULD BE DONE'', IN FACT HYPERREF NEEDS TO BE THERE OR IT WON'T WORK
% \usepackage[NoDocJS]{pdfanim}
% \usepackage{pdfanim} %requires file to be opened via open file dialogue in Acrobat Reader
% \PDFAnimLoad[remember,width=4in]{animap}{/home/ian/dogshare/dissertation/defense_talk/animap/animapB}{72}




%\usecolortheme{seahorse}
%\usecolortheme{rose}

\definecolor{section0}{rgb}{0,.5,.1} % Defines section colors!  (neat)
\definecolor{section1}{rgb}{0,.7,0}
\definecolor{section2}{rgb}{0,.5,.5}
\definecolor{Population dynamics model}{rgb}{0,.5,.4}

\definecolor{USC }{rgb}{1.0, 0.65,   0}
\definecolor{PS  }{rgb}{1.0,    0,   0}
\definecolor{SG  }{rgb}{0.8,    0, 0.8}
\definecolor{WCVI}{rgb}{0.0,    0, 0.8}
\definecolor{NBC }{rgb}{0.0,  0.8,   0}
\definecolor{titleblue}{rgb}{0.2,0.2,0.7}
% orange    #FFA500 255 165 0
% red       #FF0000 255 0   0
% magenta3  #CD00CD 205 0   205
% blue      #0000FF 0   0   255
% green3    #00CD00 0   205 0

\definecolor{green1}{rgb}{0,.5,.1}
\definecolor{cream1}{rgb}{.97,.97,.83}
\definecolor{shitbrown}{rgb}{.2,.2,.05}

	\setbeamercolor{boxcolor1}{fg=white,bg=green1}
	\setbeamercolor{boxcolor2}{fg=black,bg=cream1}


\newcommand{\Em}[1]{\emph{\color{shitbrown}{#1} }\color{black}}
\newcommand{\bl}[1]{\color{titleblue}{#1 }\color{black}}
\newcommand{\gr}[1]{\color{green1}{#1 }\color{black}}
\newcommand{\rd}[1]{\color{red}{#1 }\color{black}}
\newcommand{\bk}[1]{\color{black}{#1 }\color{black}}
\newcommand{\wh}[1]{\color{white}{#1 }\color{black}}

\newcommand{\USC }[1]{\color{USC }{#1 }\color{black}}
\newcommand{\PS  }[1]{\color{PS  }{#1 }\color{black}}
\newcommand{\SG  }[1]{\color{SG  }{#1 }\color{black}}
\newcommand{\WCVI}[1]{\color{WCVI}{#1 }\color{black}}
\newcommand{\NBC }[1]{\color{NBC }{#1 }\color{black}}


%\usepackage[graph,frame,arrow,tips,matrix]{xy}
\setbeamertemplate{items}[circle]
\setbeamertemplate{blocks}[rounded][shadow=false]
%\definecolor{emphasis}{rgb}{0,.5,.1}
%\usepackage[english]{babel}
%\usepackage[latin1]{inputenc}
%\usepackage{times}
%\usepackage[T1]{fontenc}
%\usepackage{pgf,pgfarrows,pgfnodes,pgfautomata,pgfheaps}
%\usefonttheme{structurebold}

\setbeamertemplate{items}[ball]
\mode<presentation>
  \setbeamercovered{transparent}
\usepackage{pgf}


\usetheme{Singapore}



\title{Population dynamics of spiny dogfish\\ in the Northeast Pacific}
\author{Ian Taylor}
\institute{Ph.D. Final Examination\\
Quantitative Ecology and Resource Management\\
University of Washington}
\date{April 10, 2008}


%%%%%%%%%%%%%%%%%%%%%%%%%%%%%%%%%%%%%%%%%%%%%%%%%%%%%%%%%
\begin{document}

\frame{
	\tableofcontents[part=1]
		\maketitle
}

%%%%%%
%%%%% #####################################################################
%%%%
\section{Background}
%\subsection{}


\begin{frame}
% \frametitle{Outline \movie[externalviewer]{}{MudSharkIntro-2channel-8bit-mulaw.au}}
\frametitle{Outline}
\tableofcontents%[pausesections]
% \sound[inlinesound, autostart, samplingrate=44100, channels=2, bitspersample=8, encoding=muLaw]{}{MudSharkIntro-2channel-8bit-mulaw.au}
% \hyperlinksound[inlinesound, autostart, samplingrate=44100, channels=2, bitspersample=8, encoding=muLaw]{}{MudSharkIntro-2channel-8bit-mulaw.au}
% \hyperlinksound{}{MudSharkIntro-2channel-8bit-mulaw.au}
\end{frame}

% \begin{frame}
% \frametitle{Outline}
% \begin{columns}
% 	\column{0.2\textwidth} % set width of first column
% 	\column{0.6\textwidth} % set width of second column
%   \tableofcontents
%   \column{0.2\textwidth} % set width of second column
% \end{columns}
% \end{frame}

%%%%%%%%%%%%%%%%%%%%%%%%%%%%%
%\section{Outline}
%\frame{
%			\begin{itemize}
%				\item{Movement patterns (chapter 2)}
%				\item{Demographic patterns (chapter 3)}
%				\item{Population dynamics model (chapter 4)}
%			\end{itemize}
%}

%http://maps.google.com/?ie=UTF8&ll=47.338823,-128.364258&spn=17.009754,48.164063&t=p&z=5
%http://maps.google.com/?ie=UTF8&ll=47.338823,-128.364258&spn=17.009754,48.164063&t=p&z=5&pw=2

%%%%%%%%%%%%%%%%%%%%%%%%%%%%%
\setbeamertemplate{background canvas}{\includegraphics[width=\paperwidth]{dogfish_backlight_MikeUrban_cropped.jpg}}
\begin{frame}[plain]

  \frametitle{}
  % \begin{itemize}[<->]

  % \begin{itemize}
  % \bf{\PS{\item[]Spiny dogfish (a.k.a. spurdog, cape shark, or rock salmon, \textit{Squalus acanthias})   }}
  % \bf{\PS{\item[]small, long-lived, bottom dwelling, omnivorous                                           }}
  % \bf{\PS{\item[]found in temperatue waters of all ocean basins                                           }}
  % \end{itemize}

% \sound[inlinesound, autostart, samplingrate=44100, channels=2, bitspersample=8, encoding=muLaw]{}{MudSharkMiddle-2channel-8bit-mulaw.au}
\end{frame}
\setbeamertemplate{background canvas}{} %return to blank canvas

%%%%%%%%%%%%%%%%%%%%%%%%%%%%%
\begin{frame}
	\frametitle{Spiny dogfish}
  \bl{(a.k.a. spurdog, cape shark, rock salmon, mud shark, or \textit{Squalus acanthias})}
\includegraphics[width=\textwidth]{/home/ian/pictures/dogfish_images/rangemap.jpg}
\end{frame}
\setbeamertemplate{background canvas}{} %return to blank canvas

%%%%%%%%%%%%%%%%%%%%%%%%%%%%%
\setbeamertemplate{background canvas}{\includegraphics[width=\paperwidth]{/home/ian/pictures/dogfish_images/skidegate1880.jpg}}
\begin{frame}[plain]
	\frametitle{\normalsize\PS{\hspace{1in}Skidegate, Queen Charlotte Islands, 1880}}
\end{frame}
\setbeamertemplate{background canvas}{} %return to blank canvas

%%%%%%%%%%%%%%%%%%%%%%%%%%%%%
\setbeamertemplate{background canvas}{\includegraphics[width=\paperwidth]{/home/ian/pictures/dogfish_images/RobertDavidsonDogfishPretendingToBeHumanWide.jpg}}
\begin{frame}[plain]
	\frametitle{}
  \normalsize\PS{``Dogfish pretending to be human''\\ by Robert Davidson}\vspace{2in}
\end{frame}
\setbeamertemplate{background canvas}{} %return to blank canvas

%%%%%%%%%%%%%%%%%%%%%%%%%%%%%
\setbeamertemplate{background canvas}{\includegraphics[width=\paperwidth]{/home/ian/pictures/dogfish_images/FisgardLighthouse2crop.jpg}}
\begin{frame}[plain]
	\frametitle{\normalsize\wh{Fisgard Lighthouse, Vancouver Island, built 1860, lit with dogfish oil}}
\end{frame}
\setbeamertemplate{background canvas}{} %return to blank canvas

%%%%%%%%%%%%%%%%%%%%%%%%%%%%%
\begin{frame}[plain]
\frametitle{1940s: ``spectacular'' fishery for dogfish livers}

\hspace{.2in}
\includegraphics[height=2.3in]{/home/ian/pictures/dogfish_images/trawler1947.png}
\hspace{.05in}
\includegraphics[height=2.3in]{/home/ian/pictures/dogfish_images/1940sTrawlPhotoFromKetchen.png}

% \framebox{      % just so you can see where the "picture" is
  \begin{picture}(0,0)
     \put(-30,-9){\includegraphics[width=6.12in]{/home/ian/dogshare/dissertation/ianphd/biology/figs/CatchPlotTalk.pdf}}
  \end{picture}
% }\par
\end{frame}
\setbeamertemplate{background canvas}{} %return to blank canvas


%%%%%%%%%%%%%%%%%%%%%%%%%%%%%
\setbeamertemplate{background canvas}{\includegraphics[width=\paperwidth]{/home/ian/pictures/dogfish_images/PugetSoundVillain2.pdf}}
\begin{frame}[plain]
	\frametitle{}
\end{frame}
\setbeamertemplate{background canvas}{} %return to blank canvas


%%%%%%%%%%%%%%%%%%%%%%%%%%%%%
\begin{frame}
\frametitle{Fujioka (1978) UW dissertation}
\begin{picture}(0,0)
     \put(0,0){\movie[externalviewer]{}{MudSharkIntro-2channel-8bit-mulaw.au}}
     \put(0,10){\movie[externalviewer]{}{MudSharkMiddle-2channel-8bit-mulaw.au}}
\end{picture}

\vspace{-.1in}\includegraphics[width=1.05\textwidth]{/home/ian/pictures/dogfish_images/FujiokaPg1.png}
\end{frame}

%%%%%%%%%%%%%%%%%%%%%%%%%%%%%
\begin{frame}
\frametitle{1975--today: Food}
\begin{center}
\begin{columns}
	\column{0.45\textwidth} % set width of first column
  \includegraphics[width=.9\textwidth]{/home/ian/pictures/dogfish_images/schillerlocken.jpg}
	\column{0.55\textwidth} % set width of second column
  \includegraphics[width=\textwidth]{/home/ian/pictures/dogfish_images/fishchips.jpg}
\end{columns}
\end{center}
\end{frame}

%%%%%%%%%%%%%%%%%%%%%%%%%%%%%
\begin{frame}
\frametitle{Dogfish in trouble}
  % \begin{center}
  % \hspace{-1in}\vspace{-1.5in}\fbox{ \includegraphics[width=.7\textwidth]{The_thrill_of_decline_and_the_agony_of_recovery.pdf} }%\\
  \begin{picture}(0,0)
     \put(170,-90){\fbox{ \includegraphics[width=3in]{The_thrill_of_decline_and_the_agony_of_recovery.pdf} }}
     \put(0,-100){\includegraphics[width=2in]{/home/ian/pictures/dogfish_images/IUCN2.pdf} }
     \put(0,0){\includegraphics[width=2in]{cites-banner.jpg} }

  \end{picture}

\end{frame}

%%%%%%%%%%%%%%%%%%%%%%%%%%%%%
\begin{frame}
\frametitle{Questions}
\begin{columns}
	\column{0.15\textwidth} % set width of first column

	\column{0.7\textwidth} % set width of second column

  \large{
  What was the impact of the 1940s fishery on dogfish in the NE Pacific? \\

  \bigskip
  What is the current status of dogfish in the NE Pacific? \\

  \bigskip
  What new tools can be developed for understanding population dynamics of sharks and the impacts of shark fisheries in general? \\ 

  \vspace{1in}}
  \column{0.15\textwidth} % set width of first column
\end{columns}
\end{frame}


%%%%%%%%%%%%%%%%%%%%%%%%%%%%%
\begin{frame}
\frametitle{}
\begin{center}
\begin{columns}
	\column{0.1\textwidth} % set width of first column

	\column{0.8\textwidth} % set width of second column
    \includegraphics[trim=3in 0in 0in .5in, clip=true, width=\textwidth]{/home/ian/letterhead/ian_letterhead_red_center.png}
    % \begin{itemize}
      \scriptsize{
      % Nicci, Joel, Cindy, maybe Rustin
      \textbf{Silva, A.   } 2008. Population dynamics of the blue shark, \textit{Prionace glauca}, in the North Atlantic Ocean. Ph.D. Dissertation, School of Aquatic and Fishery Sciences                                                                               \\

      \smallskip
      \textbf{Rice, J.    } 2007. A study of \PS{spiny dogfish} (\textit{Squalus acanthias}) in the Gulf of Alaska. Master's Thesis, Quantitative Ecology and Resource Management.                                                                                       \\

      \smallskip
      \textbf{Vega, N.    } 2006. Biogeography of the \PS{spiny dogfish} (\textit{Squalus acanthias}) over a latitudinal gradient in the NE Pacific. Master's Thesis, Quantitative Ecology and Resource Management.                                                      \\

      \smallskip
      \textbf{Menon, M.   } 2004. Spatial-temporal modeling of Pacific sleeper shark (\textit{Somniosus pacificus}) and \PS{spiny dogfish} (\textit{Squalus acanthias}) bycatch in the northeast Pacific Ocean. Master's Thesis, School of Aquatic and Fishery Sciences. \\

      \smallskip
      \textbf{Tribuzio, C.} 2004. An introduction of the reproductive physiology of two North Pacific shark species: \PS{spiny dogfish} (\textit{Squalus acanthias}) and salmon shark (\textit{Lamna ditropis}). Master's Thesis, School of Aquatic and Fishery Sciences \\

      \smallskip
      \textbf{Director, R.} 2001. The \PS{spiny dogfish} fishery of Puget Sound and the Pacific Northwest coast: management options and recommendations. Masters Thesis, School of Marine Affairs.                                                                       \\
      }
      % \end{itemize}
  \column{0.1\textwidth} % set width of first column
\end{columns}
\end{center}
\end{frame}

%%%%%%
%%%%% ################################ Reproductive Value #####################################
%%%%
% \setbeamercolor{normal text}{bg=red}
\setbeamertemplate{background canvas}{bg=red}
\section{Reproductive Value}
%\subsection{}

%%%%%%%%%%%%%%%%%%%%%%%%%%%%%

\setbeamertemplate{background canvas}{\includegraphics[height=\paperheight, width=\paperwidth]{/home/ian/dogshare/dissertation/ianphd/metapopulation/figs/blackSquare.pdf}}
\begin{frame}[plain]
\begin{center}
\WCVI{\Huge{Reproductive Value}}
\includegraphics[width=\textwidth]{/home/ian/pictures/dogfish_images/dogfish_art/bluedog.pdf}
\end{center}
\end{frame}
\setbeamertemplate{background canvas}{} %return to blank canvas

%%%%%%%%%%%%%%%%%%%%%%%%%%%%%
\frame{\frametitle{Problem}
\large{
The long lifespan and late age of maturity of many sharks allow harvests of different segments of the population: pups, juveniles, and adults. \\

\bigskip
How can we best quantify the impacts of these different harvests? \\ 
}

\bigskip
\hspace{.5in}\includegraphics[width=.8\textwidth]{GallucciTaylorErziniTitle.png}

% \vspace{1in}
}
%%%%%%%%%%%%%%%%%%%%%%%%%%%%%
\frame{\frametitle{Reproductive value}
Reproductive value, $v_{x}$, defined by R.A. Fisher in 1930, \\
	% $v_x  = \sum\limits_{i \ge x} {\lambda ^{ - (i - x)} f_i \frac{{l_i }}{{l_x }}}$, \\
\[v_x  = \sum\limits_{i \ge x} {\lambda ^{ - (i - x)} f_i \frac{{l_i }}{{l_x }}},\] \\
where \\
$f_i$ = fecundity at age $i$,\\
$l_i$ = probability of surviving from birth to to age $i$, and\\
$\lambda$ = annual population increase.\\

\bigskip
\textbf{\PS{Relative measure of expected future offspring for any given age}}
}

%%%%%%%%%%%%%%%%%%%%%%%%%%%%%
\frame{\frametitle{Reproductive value}
\begin{center} \includegraphics[height=2.2in]{/home/ian/dogshare/dissertation/defense_talk/RVfigure.pdf}\\
\end{center}

\vspace{-.1in}
\hrule
\vspace{.05in}
\scriptsize{
Human values from Bowles, S. and Posel D. 2005, Genetic relatedness predicts South African migrant workers' remittances to their families, \textit{Nature} 434:380--383\\ 
Dogfish values from data on Puget Sound dogfish}
}
%%%%%%%%%%%%%%%%%%%%%%%%%%%%%
\frame{\frametitle{Reproductive potential}
Define reproductive potential of the population as \\
\[P = \sum\limits_{x=0}^{N_{ages}} v_x N_x, \]
and reproductive potential removed by harvest,
\[P_h = \sum\limits_{x=0}^{N_{ages}} v_x H_x {N_x}, \]
Let $\Phi = P_h / P$ be the fraction of the population's reproductive potential harvested annually.\\

\vspace{.25in}
\hrule
\vspace{.05in}
\footnotesize{
$v_x$ is reproductive value of age $x$, $N_x$ is numbers at age $x$, and \\ $H_x$ is fraction of numbers at age $x$ removed annually.}
}
%%%%%%%%%%%%%%%%%%%%%%%%%%%%%
\frame{\frametitle{}
\vspace{.2in}
\textbf{Theorem: }(Taylor, Gallucci) The asymptotic annual growth of a
harvested population $\lambda'$ decreases linearly with $\Phi$, the fraction of the
population's reproductive potential harvested annually,
\[\lambda' = \lambda (1 - \Phi).\]

%%%%%%%%%%%%%%%%%%%%%%%%%%%%%
\begin{center} \includegraphics[height=2in]{/home/ian/dogshare/dissertation/ianphd/CJFASpaper/figs/fig3.pdf}
\end{center}
}

%%%%%%%%%%%%%%%%%%%%%%%%%%%%%
\frame{\frametitle{Reproductive value conclusions}
% \begin{itemize}
Reproductive value and reproductive potential are the appropriate measures of the impact of removing different ages from the population.

\bigskip
Theorem built on assumptions of Leslie matrix model, but results still hold in presence of variability.
\vspace{1.2in}
% \end{itemize}
}

%%%%%%
%%%%% ###################### Demography ###############################################
%%%%
\section{Demography}
%\subsection{}

% %%%%%%%%%%%%%%%%%%%%%%%%%%%%%
% \frame{\frametitle{}
% \begin{center}
% \Huge{Demography}
% \end{center}
% }



%%%%%%%%%%%%%%%%%%%%%%%%%%%%%

\setbeamertemplate{background canvas}{\includegraphics[height=\paperheight, width=\paperwidth]{/home/ian/dogshare/dissertation/ianphd/metapopulation/figs/blackSquare.pdf}}
\begin{frame}[plain]
\begin{center}
\NBC{\Huge{Demography}}
\includegraphics[width=\textwidth]{/home/ian/pictures/dogfish_images/dogfish_art/greendog.pdf}
\end{center}
\end{frame}
\setbeamertemplate{background canvas}{} %return to blank canvas

%%%%%%%%%%%%%%%%%%%%%%%%%%%%%

\begin{frame}
\frametitle{Dogfish dorsal spines from 1942--43}
\begin{center}
\begin{columns}
	\column{0.5\textwidth} % set width of first column
  \includegraphics[width=\textwidth]{spines.png}
	\column{0.5\textwidth} % set width of second column
  % Held in the UW Fish Collection. \\
  Aged using modern methods.\\

  \bigskip
  Similar data available from 2000s dogfish (Tribuzio, 2004).

  \bigskip
  Demographic parameters published for 1970s and 1980s (Ketchen, 1972; Sauders \& McFarlane, 1993).\\

  \bigskip
  \PS{Allows examination of fecundity, maturity, and growth from 1940s to 2000s.}
\end{columns}
\end{center}
\end{frame}

%%%%%%%%%%%%%%%%%%%%%%%%%%%%%
\frame{\frametitle{Changes in fecundity}
\begin{center} \includegraphics[height=3in]{/home/ian/dogshare/dissertation/ianphd/biology/figs/fecundityColor_lame.pdf}
\end{center}
}
%%%%%%%%%%%%%%%%%%%%%%%%%%%%%
\frame{\frametitle{Changes in age and length at maturity}
\begin{center} \includegraphics[height=3in]{/home/ian/dogshare/dissertation/ianphd/biology/figs/maturityColor_lame.pdf}
\end{center}
}
%%%%%%%%%%%%%%%%%%%%%%%%%%%%%
\frame{\frametitle{Changes in growth}
\begin{center} \includegraphics[height=3in]{/home/ian/dogshare/dissertation/ianphd/biology/figs/agelen50growthColor_lame.pdf}
\end{center}
}

%%%%%%%%%%%%%%%%%%%%%%%%%%%%%
\frame{
\vspace{-.3in}\begin{columns}
	\column{0.4\textwidth} % set width of first column
  \begin{flushright}
    \includegraphics[angle=0,height=1.1\textheight,width=2in]{change_comparison_plot_color_talk_vert.pdf}
  \end{flushright}
	\column{0.6\textwidth} % set width of second column
  \Large{\bl{Climate or fishing?}}\\

  \normalsize
  \bigskip
  Intermediate results from 1970s/1980s can be used to estimate timing of change

  \bigskip
  \wh{Greatest change occurred between 1940s and 1970s/1980s when temperature was stable}

  \bigskip
  \wh{Fishing most likely cause of change}

\end{columns}

}
%%%%%%%%%%%%%%%%%%%%%%%%%%%%%
\frame{
\vspace{-.3in}\begin{columns}
	\column{0.4\textwidth} % set width of first column
  \begin{flushright}
    \includegraphics[angle=0,height=1.1\textheight,width=2in]{change_comparison_plot_color_talk_vert_bold.pdf}
  \end{flushright}
	\column{0.6\textwidth} % set width of second column
  \Large{\bl{Climate or fishing?}}\\

  \normalsize
  \bigskip
  Intermediate results from 1970s/1980s can be used to estimate timing of change

  \bigskip
  Greatest change occurred between 1940s and 1970s/1980s when temperature was stable

  \bigskip
  \PS{Fishing} most likely cause of change

\end{columns}

}

%%%%%%%%%%%%%%%%%%%%%%%%%%%%%
\frame{\frametitle{Demography conclusions}

\begin{columns}
	\column{0.2\textwidth} % set width of first column

	\column{0.6\textwidth} % set width of second column

  Dogfish demography is remarkable.

  \bigskip
  Fecundity, maturity, and growth changed significantly between 1940s and 2000s.

  \bigskip
  Changes in density appear more likely as \\the cause than changes in climate.
  \vspace{1.2in}
  \column{0.1\textwidth} % set width of first column
\end{columns}


%%%%%%
%%%%% ##################### Population Dynamics ################################################
%%%%
\section{Population Dynamics}

%%%%%%%%%%%%%%%%%%%%%%%%%%%%%

\setbeamertemplate{background canvas}{\includegraphics[height=\paperheight, width=\paperwidth]{/home/ian/dogshare/dissertation/ianphd/metapopulation/figs/blackSquare.pdf}}
\begin{frame}[plain]
\begin{center}
\PS{\Huge{Population Dynamics}}
\includegraphics[width=\textwidth]{/home/ian/pictures/dogfish_images/dogfish_art/reddog.pdf}
\end{center}
\end{frame}
\setbeamertemplate{background canvas}{} %return to blank canvas

%%%%%%%%%%%%%%%%%%%%%%%%%%%%%
\frame{\frametitle{Metapopulation model with 5 areas}
\framesubtitle{Integrated model fit to data on abundance, length and age compositions, and tag recaptures}
\begin{center}
\includegraphics[width=.95\textwidth]{/home/ian/dogshare/dissertation/ianphd/metapopulation/figs/eezmapColor.pdf}
\end{center}
}

%%%%%%%%%%%%%%%%%%%%%%%%%%%%%
\frame{\frametitle{}
\begin{center}
\includegraphics[height=\textheight]{/home/ian/dogshare/dissertation/ianphd/metapopulation/figs/CatchTagPlotColor.pdf}
\end{center}
}

%%%%%%%%%%%%%%%%%%%%%%%%%%%%%
\frame{
\frametitle{Model structure}
\framesubtitle{122 ages $\times$ 72 years $\times$ 5 areas =  43,920 bins for numbers at age}
\begin{center}
\vspace{-.2in}\includegraphics[height=2.8in]{/home/ian/dogshare/dissertation/ianphd/metapopulation/figs/GridFigColor.pdf}
\end{center}
}
%%%%%%%%%%%%%%%%%%%%%%%%%%%%%
\frame{
\frametitle{Model flow}
\begin{center}
  \vspace{-.3in}\includegraphics[height=3.0in, trim=1.3in 4in 1.0in 1in, clip=true]{/home/ian/dogshare/dissertation/ianphd/metapopulation/figs/flowchart/flowchart_expanded_fixed_color.pdf}
\end{center}
}

\frame{\frametitle{Outputs from model}

\begin{columns}
	\column{0.2\textwidth} % set width of first column

	\column{0.8\textwidth} % set width of second column

  Estimates (with quantified uncertainty) for:\\
  \begin{itemize}
    \item equilibrium (1935) abundance in each area
    \item current (2006) abundance in each area
    \item what happened in between
    \item movement rates between adjacent areas
  \end{itemize}
  Fit of model to observed values from:
  \begin{itemize}
    \item indices of abundance (CPUE)
    \item age and length compositions
    \item recaptures of tagged dogfish
  \end{itemize}

  % \vspace{1.0in}
  % \column{0.1\textwidth} % set width of first column
\end{columns}
}

%%%%%%%%%%%%%%%%%%%%%%%%%%%%
\frame{\frametitle{\hspace{.2in}Results summary}
\begin{center}
\vspace{-.9in}\includegraphics[height=3.3in]{/home/ian/dogshare/dissertation/ianphd/metapopulation/figs/animap1935.pdf}
\vspace{-.9in}\includegraphics[height=3.3in]{/home/ian/dogshare/dissertation/ianphd/metapopulation/figs/animapEmpty.pdf}
\end{center}
}
%%%%%%%%%%%%%%%%%%%%%%%%%%%%
\frame{\frametitle{\hspace{.2in}Results summary}
\begin{center}
\vspace{-.9in}\includegraphics[height=3.3in]{/home/ian/dogshare/dissertation/ianphd/metapopulation/figs/animap1935.pdf}
\vspace{-.9in}\includegraphics[height=3.3in]{/home/ian/dogshare/dissertation/ianphd/metapopulation/figs/animap2006.pdf}
\end{center}
}
%%%%%%%%%%%%%%%%%%%%%%%%%%%%%
\frame{\frametitle{Biomass trajectories}
\begin{center}
\vspace{-.5in}
\includegraphics[height=.95\textheight, width=\textwidth]{/home/ian/dogshare/dissertation/ianphd/metapopulation/figs/BiomassTrajectoriesHorizontalColor.pdf}
\end{center}
}
%%%%%%%%%%%%%%%%%%%%%%%%%%%%
\frame{\frametitle{Fit to indices of abundance}
\begin{center}
\vspace{-.1in}
\includegraphics[height=.9\textheight]{/home/ian/dogshare/dissertation/ianphd/metapopulation/figs/IndexFitsHorizontalColor.pdf}
\end{center}
}
%%%%%%%%%%%%%%%%%%%%%%%%%%%%%
\frame{\frametitle{\hspace{3.5in}Length and age\\ \hspace{3.4in}compositions}
\vspace{-.6in}\includegraphics[height=\textheight]{/home/ian/dogshare/dissertation/ianphd/metapopulation/figs/CompsColor.pdf}
}
%%%%%%%%%%%%%%%%%%%%%%%%%%%%%
\frame{\frametitle{\hspace{3.5in}Tag \\ \hspace{3.8in}recaptures}
\vspace{-.6in}\includegraphics[height=\textheight]{/home/ian/dogshare/dissertation/ianphd/metapopulation/figs/tagproportionsSidewaysColorFixed.pdf}
}
%%%%%%%%%%%%%%%%%%%%%%%%%%%%%
\frame{\frametitle{Posterior distributions for initial biomass}
\includegraphics[width=\textwidth]{/home/ian/dogshare/dissertation/ianphd/metapopulation/figs/B0postColor.pdf}
}

%%%%%%%%%%%%%%%%%%%%%%%%%%%%%
\frame{\frametitle{\hspace{.2in}Sensitivities}
\begin{center}
\vspace{-.7in}\includegraphics[height=3.0in]{/home/ian/dogshare/dissertation/ianphd/metapopulation/figs/SensTraject.pdf}
\vspace{-.7in}\includegraphics[height=3.0in]{/home/ian/dogshare/dissertation/ianphd/metapopulation/figs/SensTraject2.pdf}
\end{center}
}
%%%%%%%%%%%%%%%%%%%%%%%%%%%%%
\frame{\frametitle{Population dynamics conclusions}
% \begin{itemize}

\begin{columns}
	\column{0.15\textwidth} % set width of first column

	\column{0.7\textwidth} % set width of second column

  Metapopulation framework valuable in estimating dogfish abundance and movement rates

  \bigskip
  Coastal areas are tightly connected---less connectivity with Puget Sound and Strait of Georgia

  \bigskip
  Initial biomass estimates sensitive to assumptions

  \bigskip
  Biomass most likely above 40\% of initial level

  \bigskip
  Evidence for additional factors not included in model

  \column{0.15\textwidth} % set width of first column
\end{columns}

}

%%%%%%
%%%%% ################################ Conclusions #####################################
%%%%
\section{Conclusions}
%\subsection{}


\frame{\frametitle{General conclusions}

\begin{columns}
	\column{0.15\textwidth} % set width of first column

	\column{0.7\textwidth} % set width of second column

  Dogfish are a worthy subject for study

  \bigskip
  Reproductive value and reproductive potential give insight into fishing impacts

  \bigskip
  Samples spanning long time-periods allow important demographic comparisons

  \bigskip
  Spatial metapopulation models are extremely valuable to understanding population dynamics

  \column{0.15\textwidth} % set width of first column
\end{columns}

}

%%%%%%%%%%%%%%%%%%%%%%%%%%%%%
\begin{frame}[shrink=20]
\frametitle{Acknowledgements}
  \begin{picture}(0,0)
     \put(240,-270){\includegraphics[width=2in]{/home/ian/pictures/pupheads2.png}}
  \end{picture}
\footnotesize{
  \begin{columns}
    \column{0.27\textwidth} % set width of first column
    \begin{itemize}
      \USC{\item[] Committee}
      \begin{itemize}
        \item[] \underline{Vince Gallucci}
        \item[] Andr\'e Punt
        \item[] Rick Methot
        \item[] Sandy McFarlane
        \item[] Bob Gara
      \end{itemize}

      \PS{\item[] Faculty and staff}
      \begin{itemize}
        \item[] Loveday Conquest
        \item[] Mark Kot
        \item[] David Ford
        \item[] Joanne Besch
        \item[] Scott Schafer
        \item[] April Wilkinson
        \item[] Hyun Kyoung
      \end{itemize}
      % \vspace{1in}

    \end{itemize}
    \column{0.38\textwidth} % set width of 2nd column
    \begin{itemize}

      \SG{\item[] Agency folks: WDFW, NMFS, and DFO}
      \begin{itemize}
        \item[] Greg Lippert,
                Theresa Tsou,
                Greg Bargmann,
                Ian Stewart,
                Kevin Piner,
                Mark Wilkins,
                John Wallace,
                Stacey Miller,
                Vanessa Tuttle,
                Jackie King,
                Jeff Fargo
                Bill Clark
      \end{itemize}

      \WCVI{\item[] Shark lab people}
      \begin{itemize}
        \item[] Alex,
                Joel,
                Hans,
                Shannon,
                Nicci,
                Muktha,
                Cindy,
                Jason,
                Pilar,
                Brian,
                Danny,
                Ben,
                Miguel
      \end{itemize}
          \NBC{\item[] Groups}
      \begin{itemize}
        \item[] NMFS for funding
        \item[] UAW Local 4121
        \item[] countless open source \\ developers
      \end{itemize}
    \end{itemize}
      \column{0.45\textwidth} % set width of 2nd column
    \begin{itemize}

      \USC{\item[] Fellow grads}
      \begin{itemize}
        \item[]
                Aditya,
                Allan,
                Bert,
                Bob,
                Carey,
                Eli,
                Dawn,
                Derek,
                Gavin,
                Ian,
                James,
                Jason,
                Jenn,
                John,
                Jon,
                Jo,
                Juan,
                Julian,
                Kevin,
                Kristin,
                Melissa,
                Mike,
                Nathan,
                Nathalie,
                Nicolas,
                Rebecca,
                Rishi,
                Saang-Yoon,
                Stephani,
                Teresa,
                Ting,
                Trevor,
                and many others
      \end{itemize}
    \PS{\item[] Friends and family}
      \begin{itemize}
        \item[] Calin,
                Dan, Janet, Nat, Karissa,
                Liz, Marsh,
                Dave,
                Katie,
                Mario,
                Mia,
                Laela,
                Travis,
                Erin,
                Erik,
                Sonja,
                Laura,
                Robert,
                Kate,
		Deane,
                Nora,
                Spackle,
                Hoover
      \end{itemize}
    \end{itemize}
  \end{columns}
  } % end \footnotesize

\end{frame}


\end{document}

